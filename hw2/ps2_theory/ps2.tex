\documentclass[11pt,english]{article}

%%%%%%%%%%%%%%%%%%%%%%%%%%%%%%%%%%%%%%%%%%%%%%%%%%%%%%%%%%%
% Packages
%%%%%%%%%%%%%%%%%%%%%%%%%%%%%%%%%%%%%%%%%%%%%%%%%%%%%%%%%%%

% Package imports are stored in /assets/base-packages.tex
% Packages specific to this pset can be imported here.

\listfiles
\input{assets/base-packages.tex}

%%%%%%%%%%%%%%%%%%%%%%%%%%%%%%%%%%%%%%%%%%%%%%%%%%%%%%%%%%%
% Shortcuts
%%%%%%%%%%%%%%%%%%%%%%%%%%%%%%%%%%%%%%%%%%%%%%%%%%%%%%%%%%%
\usepackage{assets/mysymbols}
\input{assets/math_definition}
\renewcommand{\hide}[1]{}

%%%%%%%%%%%%%%%%%%%%%%%%%%%%%%%%%%%%%%%%%%%%%%%%%%%%%%%%%%%
% Title / Author
%%%%%%%%%%%%%%%%%%%%%%%%%%%%%%%%%%%%%%%%%%%%%%%%%%%%%%%%%%%
\begin{document}
\title{CS 4644/7643: Deep Learning\\
Spring 2025 \\
Problem Set 2} 

% NOTE: Any changes to instructor, TAs, or piazza link should be done in the file below
\author{Instructor: Zsolt Kira \\
TAs: Wei Zhou, Aryan Sarswat, Woo Chul Shin, Elias Cho, \\Haotian Xue, Sri Siddarth Chakaravarthy Prakash, Neelabh Sinha, David He, \\ Sriharsha Kocherla, Pratham Mehta, Ayush Patel, Cari He\\
Discussions: \url{https://piazza.com/class/m5k29i4gzsf4ab/}}
\date{Deadline: 11:59pm February 17st, 2025}
\maketitle


\paragraph*{Instructions}
\begin{enumerate}
\item We will be using Gradescope to collect your assignments.  Please read the following instructions for submitting to Gradescope carefully!
     \begin{itemize}
          \item
               Please upload separate pdfs for \textbf{HW2 Theory} and \textbf{HW2 Report} sections on Gradescope. The instructions for the latter are included in the coding section of the assignment.  \textbf{When submitting to Gradescope, please make sure to mark the page(s) corresponding to each problem/sub-problem}.
          \item
               For the \textbf{HW2 Code Part 1 and Part 2} components on Gradescope, 
               please use the \texttt{collect\_submission.py} script provided and upload the resulting zip files to each Gradescope section respectively. Please make sure you have saved the most recent version of your code.
          \item
               Note: This is a large class and Gradescope's assignment segmentation features are essential.
               Failure to follow these instructions may result in parts of your assignment not being graded.
               We will not entertain regrading requests for failure to follow instructions.
     \end{itemize}

\item
     \LaTeX'd solutions are strongly encouraged (solution template available in the zip file in HW2 under the Assignments tab on Canvas),
     but scanned handwritten copies are acceptable.
     Hard copies are \textbf{not} accepted.


\item We generally encourage you to collaborate with other students.

You may talk to a friend, discuss the questions and potential directions for solving them. However, you need to write
your own solutions and code separately, and \emph{not} as a group activity.
Please list the students you collaborated with. \\ \\
\end{enumerate}
\newpage

%%%%%%%%%%%%%%%%%%%%%%%%%%%%%%%%%%%%%%%%%%%%%%%%%%%%%%%%%%%
% Body
%%%%%%%%%%%%%%%%%%%%%%%%%%%%%%%%%%%%%%%%%%%%%%%%%%%%%%%%%%%

\section{Collaborators [0.5 points]}

\input{assets/collaborators}

% This is for next year (2026)
\section{Activation Function}
\begin{enumerate}[start]

\item
\textbf{[2 points]}

In neural networks, activation functions introduce non-linearities, enabling the network to approximate complex functions. One of the desirable properties for an activation function is to be \emph{zero-centered}. Being zero-centered helps in achieving faster convergence during training, as the weights can adjust in both positive and negative directions more efficiently. Additionally, zero-centered functions can help mitigate the vanishing gradient problem, ensuring that gradients during backpropagation do not diminish too quickly.

Definition: A function \( g(x) \) is said to be zero-centered if, for every value \( x \) in its domain where \( g(x) \) is positive, there exists an equivalent negative value \( -x \) such that:
\[
  g(-x) = -g(x).
\]
In other words, the function is symmetric about the origin, producing positive outputs for positive inputs and negative outputs for negative inputs.

Consider the following activation function, defined for all real \(x\):
\[
  g(x) \;=\; \frac{x}{1 + |x|}.
\]

\begin{enumerate}
  \item Zero-Centered Property.
  Show that \( g(x) \) is zero-centered by proving \( g(-x) = -g(x) \) for all \( x \). 
  (Hint: carefully handle the absolute value.)

  \item Derivative and Gradient Behavior
    \begin{enumerate}
      \item Compute the derivative \( g'(x) \) for \( x \neq 0 \).
      \item Evaluate \( g'(0) \) (Hint: the function is continuous).
      \item Based on your results, show whether \( g(x) \) might cause vanishing or exploding gradients for large values of \( |x| \) (Hint: show the limits).
    \end{enumerate}
\end{enumerate}

\end{enumerate}

\section{Gradient Descent}
\begin{enumerate}[start]

\item
\textbf{[3 points]}
\input{assets/gradient_descent_update_rule}

\item
\textbf{[Extra Credit - 2 points]}
\input{assets/gradient_descent_convergence_rate_1}

\item
\textbf{[Extra Credit - 2 points]}
\input{assets/gradient_descent_convergence_rate_2}

\end{enumerate}


\section{Automatic Differentiation}
\begin{enumerate}[resume]

\item
\textbf{[4 points]}
\input{assets/automatic_differentiation}

\end{enumerate}

\section{Convolutions}
\begin{enumerate}[resume]

\item 
\textbf{[5 points]}
\input{assets/circulant_matrices}

\end{enumerate}


\section{SGD}
\begin{enumerate}[resume]

\item 
\textbf{[3 points]}
\input{assets/sgd_decrease_loss}

\end{enumerate}


\section{Paper Review}
\begin{enumerate}[resume]

\item 
\textbf{[4 points]}
\input{assets/ICLR_2017_rethinking_generalization}

\end{enumerate} 


\end{document}
